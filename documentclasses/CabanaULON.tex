\documentclass{CabanaSOP}

\title{Cabana Unattended Laboratory Operation Notices (ULONs)}
\author{Michael Plews}
\date{2018-01-07}

\begin{document}

\maketitle

This document describes the use of the \texttt{CabanaULON} class for
creating unattended laboratory operation notices for the Cabana
research group at the University of Illinois Chicago. It provides a
formatted layout with pre-defined tables.

\section*{Example}

All commands for setting the variables are functional. A full list of
commands can be found at the end of this document.

\begin{verbatim}
\documentclass{CabanaULON}

%Reusable information such as \Name,
%\PI can be input from a reusable
%source
\input{info.txt}

\Chemicals{
  Lithium Fluoride \newline
  Iron (III) Fluoride \newline
  Carbon
}
\Room{4163SES}
\Location{Tube Furnace}
\Start{\today}
\End{\tomorrow}

%% Hazards must be marked as {\checked}
\Biohazard{\checked}
\XRay{\checked}

\begin{document}
\makeulon 
\end{document}
\end{verbatim}

\section*{Command List}
\subsection{User Information}
\begin{verbatim}
%Define the group Chemical Hygiene Officer
\CHO{}

%Your name
\Name{}

%Your contact number
\ContactNumber{}

%Lab PI
\PI{}

%Lab PI's Number
\PINumber{}
\end{verbatim}

\subsection{Experiment Information}
\begin{verbatim}
%Experiment start date 
\Start{}

%Experiment start time
\StartTime{}

%Experiment end date
\End{}

%Experiment end time
\EndTime{}

%Experiment description
\Description{}

%Room number of experiment
\Room{}

%Location within room 
\Location{}

%Chemicals used in synthesis 
\Chemicals{}

%Emergency Shutdown procedure
\EmergencyShutdown{}
\end{verbatim}

\end{document}
