\documentclass[aspectratio=149]{beamer}
\usepackage[utf8]{inputenc}
\usepackage[T1]{fontenc}

\usetheme{anl}

\title{There Is No Largest Prime Number}
\subtitle{But they do get pretty big\ldots}
\date[ISPN ’80]{27th International Symposium of Prime Numbers}
\author[Euclid]{Euclid of Alexandria \texttt{euclid@anl.gov}}
\coverphoto{graphics/anl-aerial.jpg}

%% Used guidance from: https://tex.stackexchange.com/questions/146529/design-a-custom-beamer-theme-from-scratch

\begin{document}

\begin{frame}
\titlepage
\end{frame}

\begin{frame}
  \frametitle{There Is No Largest Prime Number}
  \framesubtitle{The proof uses \textit{reductio ad absurdum}.}
  \begin{theorem}
    There is no largest prime number.
  \end{theorem}
  \begin{proof}
    \begin{enumerate}
    \item<1-| alert@1> Suppose $p$ were the largest prime number.
    \item<2-> Let $q$ be the product of the first $p$ numbers.
    \item<3-> Then $q+1$ is not divisible by any of them.
    \item<1-> But $q + 1$ is greater than $1$, thus divisible by some primenumber not in the first $p$ numbers.
      \qedhere
    \end{enumerate}
  \end{proof}
\end{frame}

\section{The Proof}

\begin{frame}{}
\tableofcontents
\end{frame}

\begin{frame}
  \frametitle{Implications Of Our Theorem}
  \begin{block}{Counting Primes}
    No matter which prime you have, you can find a bigger one.
  \end{block}
  \begin{exampleblock}{How We Count Primes}
    3 is prime, but so is 7 and 7 is bigger than 3.
  \end{exampleblock}
  \begin{alertblock}{Warning!}
    Thinking you have the biggest prime can lead to poor results:
    \begin{itemize}
    \item Embarrassment
    \item Wrong results
    \item Broken encryption
    \end{itemize}
  \end{alertblock}
\end{frame}

\begin{frame}{Next steps}
  \begin{itemize}
  \item Look for some big prime numbers.
  \item Then look for a bigger one.
  \end{itemize}
\end{frame}

\end{document}
