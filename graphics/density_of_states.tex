\documentclass[crop,tikz]{standalone}
% \usepackage{tikz}
\usepackage{tikzscale}
\usepackage{mhchem}
\usepackage{pgf}
\usetikzlibrary{
  intersections,
  matrix,
  positioning,
  calc,
  decorations.pathmorphing
}
\usepackage{pgfplots}
\usepgfplotslibrary{
  fillbetween
}

\begin{document}
\definecolor{airforce}{HTML}{008000}
\begin{tikzpicture}
    %% Li0
    \node at (0,0) (a) {};
    \draw[->,>=latex] (0,0) -- ++(0,4) node[anchor=south] (atop) {$E$};
    \draw[] (0,0) -- node[anchor=north] (label) {\ce{LiCoO2}} ++(2,0);
    \draw[fill=blue, opacity=0.5, draw=none, thick] (0,0) -- node[anchor=west, text opacity=1] (2p) {} (0,1) arc(90:0:1) --(0,0) --cycle;
    \draw[fill=none, draw=blue, thick] (0,0) -- (0,1) arc(90:0:1) --(0,0) --cycle;
    \draw[fill=none, draw=none, thick, name path=p-M2p-L] (0, 0.9) -- node[anchor=west] (3d) {} (0,1.4) arc(90:-90:1 and 0.25) --(0,0.9);
    \draw[fill=none,draw=none,name path=fermi-L] (0,0.9) rectangle (1, 1.15) node[anchor=west] (Ef) {};
    \draw[thick, dashed, overlay] (Ef.center) -- ++(0.5,0) node[anchor=west] () {E$_{f(C)}$};
    
    \fill [airforce,
      opacity=0.5,
        intersection segments={
        of=p-M2p-L and fermi-L,
        sequence={A0--B1--A3}
                  }];
    \draw[fill=none, draw=airforce, thick, name path=p-M2p-L] (0, 0.9) -- (0,1.4) arc(90:-90:1 and 0.25) --(0,0.9);
    \draw[fill=none, draw=red, thick] (0,2.5) -- node[anchor=west] (4s) {} (0,3.5) arc(90:-90:0.5 and 0.5) --(0,2.5);
    \begin{scope}
      \clip (4s.west) rectangle ++(-1.25, 0.65) node[anchor=east] (Li) {};
      \draw[fill=orange,opacity=0.5,draw=none, thick] (4s.west) arc(270:90:0.75 and 1) --cycle;
    \end{scope}
    \draw[fill=none,draw=orange, thick] (4s.west) arc(270:180:0.75 and 1);
    \draw[thick, dashed, overlay] (Li.center) ++(1.5,0) -- ++(1.5,0) node[anchor=west] () {E$_{f(A)}$};
    \node at (Li.center) {Li $2s$};

    \node[right=8 of 4s, anchor=west] () {O $3s$};
    \node[right=8 of 2p, anchor=west] () {O $2p$};
    \node[right=8 of 3d, anchor=west] () {M $3d$};

    %% Li0.5
    \node[right=3 of a.center] (b) {};
    \draw[->,>=latex] (b.center) -- ++(0,4) node[anchor=south] (btop) {$E$};
    \draw[] (b.center) -- node[anchor=north] (label) {\ce{Li_{0.5}CoO2}} ++(2,0);
    \draw[fill=blue, opacity=0.5, draw=none, thick] (b.center) -- ++(0,1) arc(90:0:1) -- (b.center);
    \draw[fill=none, draw=blue, thick] (b.center) -- ++(0,1) arc(90:0:1) --(b.center) --cycle;
    \draw[fill=none, draw=none, thick, name path=p-M2p-L] (b) ++(0, 0.9) node (b1) {}-- ++(0,0.5) arc(90:-90:1 and 0.25) --(b1);
    \draw[fill=none,draw=none,name path=fermi-L] (b1) rectangle ++(1, 0.125);

    \fill [airforce,
      opacity=0.5,
       intersection segments={
        of=p-M2p-L and fermi-L,
        sequence={A0--B1--A3}
                  }];
    \draw[fill=none, draw=airforce, thick, name path=p-M2p-L] (b) ++(0, 0.9) node (b1) {}-- ++(0,0.5) arc(90:-90:1 and 0.25) --(b1);
    \draw[fill=none, draw=red, thick] (b) ++(0,2.5) -- ++(0,1) arc(90:-90:0.5 and 0.5) -- ++(0,1);

    %% Li1
    \node[right=3 of b.center] (c) {};
    \draw[->,>=latex] (c.center) -- ++(0,4) node[anchor=south] (ctop) {$E$};
    \draw[] (c.center) -- node[anchor=north] (label) {\ce{Li0CoO2}} ++(2,0);
    \begin{scope}
      \clip (c) rectangle ++(1, 0.9);
      %% \draw[red] (c) rectangle ++(1, 0.9);
      \fill[fill=blue, opacity=0.5, draw=none, thick, name path=big blue] (c.center) -- ++(0,1) arc(90:0:1) -- (c.center);
      %% \fill[red!50!blue, overlay] (13mm, 0) circle;
    \end{scope}
    \draw[fill=none, draw=blue, thick] (c.center) -- ++(0,1) arc(90:0:1) --(c.center) --cycle;
    \draw[fill=none, draw=none, thick, name path=p-M2p-L] (c) ++(0, 0.9) node (c1) {}-- ++(0,0.5) arc(90:-90:1 and 0.25) --(c1);
    \draw[fill=none,draw=none,name path=fermi-L] (c1) ++(0,-0.24) rectangle ++(1, 0.25);

    \draw[fill=none, draw=airforce, thick, name path=p-M2p-L] (c) ++(0, 0.9) node (c1) {}-- ++(0,0.5) arc(90:-90:1 and 0.25) --(c1);
    \draw[fill=none, draw=red, thick] (c) ++(0,2.5) -- ++(0,1) arc(90:-90:0.5 and 0.5) -- ++(0,1);
  \end{tikzpicture}
\end{document}